\documentclass[11pt]{article}
\usepackage{amsmath}
\usepackage{multicol}
\input longdiv.tex
\usepackage{polynom}
\usepackage{tikz}
\usetikzlibrary{trees}

\topmargin -.5in
\textheight 9in
\oddsidemargin -.25in
\evensidemargin -.25in
\textwidth 7in

\begin{document}

\tikzstyle{level 1}=[level distance=3cm, sibling distance=4cm]
\tikzstyle{level 2}=[level distance=3cm, sibling distance=2cm]
\tikzstyle{level 3}=[level distance=3cm, sibling distance=1cm]


% Define styles for bags and leafs
\tikzstyle{bag} = [text width=4em, text centered]
\tikzstyle{end} = [circle, minimum width=3pt,fill, inner sep=0pt]

\author{Tom Pywell}
\title{2019 Junior Certificate Higher Level Paper Two}
\maketitle

\medskip

\begin{enumerate}
    %question 1
    \pagebreak
    \item
        \begin{enumerate}
            \item The range of data is the difference between the smallest and largest values.
                \begin{equation*}
                    172-141=31
                \end{equation*}
            \item
                An outlier is a piece of data which 'stands out' for being different.
                Usually a value much larger or smaller than the rest.
                \begin{equation*}
                    141\text{ is much smaller than the rest}
                \end{equation*}
            \item
                The mean is what we generally consider as the 'average' of the data.
                \begin{equation*}
                    \frac{141+165+167+168+169+170+172+172}{8}=165.5cm
                \end{equation*}
            \item
                For most of the teachers to be taller than 175cm, then the median height must be 175cm or greater.
        \end{enumerate}
    
    %question 2
    \pagebreak
    \item
        \begin{enumerate}
            \item Categorical Nominal\\
                
            \item Car\\
            
            \item
            \begin{center}
                \begin{tabular}{ |c|c|c|c|c| } 
                    \hline
                    &\multicolumn{2}{c|}{2006}&\multicolumn{2}{c|}{2016}\\
                    \hline
                    \text{way of travelling}&men&women&men&women\\
                    \hline
                    \text{walk or cycle}&12.1&15.4&12.2&12.6\\ 
                    \text{bus or train}&8.0&11.3&8.3&10.5\\ 
                    \text{car}&62.6&71.4&60.0&71.9\\
                    \text{other}&15.3&0.7&14.2&0.4\\
                    \text{not stated}&2&1.2&5.3&4.6\\
                    \hline
                    \text{Total}&100.0&100.0&100.0&100.0\\
                    \hline
                \end{tabular}
           \end{center}
           
            \item 
                \begin{equation*}
                        15.4 > 12.1
                \end{equation*}

            \item Alison is incorrect. If more women were surveyed in 2006, even a larger percentage in 2016 could repersent a smaller amount.

            \item
                \begin{equation*}
                    \begin{split}
                        990000*8.3\%+880000*10.5\%&=174570\text{ people who travelled to work by bus or train}\\
                        \frac{174570}{990000+880000}*\frac{100}{1}&=9.34\%
                    \end{split}
                \end{equation*}

            \item
                \begin{enumerate}
                    \item
                        \begin{center}
                            \begin{tabular}{ |c|c|c|c|c|c| } 
                                \hline
                                way of travelling&walk or cycle&bus or train&car&other&not stated\\
                                \hline
                                \% of men in 2016&12.2&8.3&60.0&14.2&5.3\\
                                \hline
                                Angle in pie chart&44&30&216&51&19\\
                                \hline
                            \end{tabular}
                    \end{center}
                        \begin{equation*}
                            \begin{split}
                                360*12.2\%&=44\\
                                360*8.3\%&=30\\
                                360*60\%&=216\\
                                360*14.2\%&=51\\
                                360*5.3\%&=19
                            \end{split}
                        \end{equation*}
                    \item TODO
                \end{enumerate}
                \item The amount of students travelling to school by foot or bicycle saw a large decline between 1986 and 2002, from about 50\% to 28\% but has remained steady since.
                    The amount of travelling by car showed the opposite, a large increase, starting at 25\% and ending at close to 60\%.
                    The amount of students taking the bus or train shows a much less aggresive decline since 1986.
            \end{enumerate}
            
    %question 3
    \pagebreak
    \item
        \begin{center}
            \begin{tabular}{|c|c|} 
                \hline
                Transformation&Image\\
                \hline
                Axial Symmetry in the x-axis&S\\
                Axial Symmetry in the y-axis&T\\
                Translation&P\\
                Central symmetry in the origin&Q\\
                \hline
            \end{tabular}
        \end{center}
            

    %question 4
    \pagebreak
    \item
        \begin{enumerate}
            \item TODO
            \begin{equation*}
                \begin{split}
                \end{split}
            \end{equation*}
            
            \item TODO
            \begin{equation*}
                \begin{split}
                \end{split}
            \end{equation*}
        \end{enumerate}

    %question 5
    \pagebreak
    \item
        \begin{enumerate}
            \item 
            \begin{equation*}
                \begin{split}
                    \text{Shop }&(-3, 1)\\
                    \text{Home }&(4, 2)\\
                    \text{School }&(4, 5)
                \end{split}
            \end{equation*}
            
            \item Use the midpoint formula to find the midpoint of 2 points, (x1, y1) and (x2, y2).
            \begin{equation*}
                \begin{split}
                    Home&=(4,2)=(x_{1},y_{1})\\
                    Shop&=(-3,1)=(x_{2},y_{2})\\
                    midpoint&=(\frac{x_{1}+x_{2}}{2} , \frac{y_{1}+y_{2}}{2})\\
                    &=(\frac{4+-3}{2} , \frac{2+1}{2})\\
                    &=(\frac{1}{2} , \frac{3}{2})\\
                    &=(0.5,1.5)
                \end{split}
            \end{equation*}

            \item Use the distance formula to find the distance between 2 points, (x1, y1) and (x2, y2).
            \begin{equation*}
                \begin{split}
                    Home&=(4,2)=(x_{1},y_{1})\\
                    Shop&=(-3,1)=(x_{2},y_{2})\\
                    distance &= \sqrt{\left({x_1-x_2}\right)^2+\left({y_1-y_2}\right)^2}\\
                    &=\sqrt{\left({4--3}\right)^2+\left({2-1}\right)^2}\\
                    &=\sqrt{\left({7}\right)^2+\left({1}\right)^2}\\
                    &=\sqrt{49+1}\\
                    &=\sqrt{50}\\
                    &=7.07\\
                \end{split}
            \end{equation*}

            \item
            \begin{equation*}
                \begin{split}
                    \text{1 unit on diagram}&=\text{2500 units in real life.}\\
                    \text{8.1 cm on diagram}&=\text{2500*8.1cm in real life}\\
                    &=20250cm\\
                    &=202.5m\\
                \end{split}
            \end{equation*}

            \item It is very unlikely she walks an exact straight line between the shop and the school. Any deviation from the straight line willl result in a larger distance.

            \item Use the slope formula to find the slope between 2 points, (x1, y1) and (x2, y2).
            \begin{equation*}
                \begin{split}
                    Shop&=(-3,1)=(x_{1},y_{1})\\
                    School&=(4,5)=(x_{2},y_{2})\\
                    slope&=(\frac{y_{2}-y_{1}}{x_{2}-x_{1}})\\
                    &=(\frac{5-1}{4--3})\\
                    &=(\frac{4}{7})\\
                \end{split}
            \end{equation*}

            \item Use the equation of a line formula with slope m and any point (x1, y1).
            \begin{equation*}
                \begin{split}
                    slope &= m = \frac{4}{7}\\
                    point &= (x_{1},y_{1}) = (4,5)\\
                    y-y_{1}&=m(x-x_{1})\\
                    y-5&=\frac{4}{7}(x-4)\\
                    7y-35&=4x-16\\
                    4x-7y+19&=0
                \end{split}
            \end{equation*}

            \item If we make a triangle with 
            \begin{equation*}
                \begin{split}
                    tan(l)&=\frac{4}{7}\\
                    tan^{-1}(\frac{4}{7})&=l\\
                    29.7\ degrees=l
                \end{split}
            \end{equation*}
        \end{enumerate}

    %question 6
    \pagebreak
    \item
        \begin{enumerate}
            \item \ 
            \begin{center}
                \begin{tabular}{|c|c|c|c|} 
                    \hline
                    Type of angle&Acute&Reflex&Obtuse\\
                    \hline
                    Angle&K&N&M\\
                    \hline
                \end{tabular}
            \end{center}
            
            \item
            \begin{equation*}
                \begin{split}
                    K&=\text{60 degrees, it's an equilateral}\\
                    M&=\text{120 degrees, because K+M=180 degress (straight line)}\\
                    N&=\text{240 degrees, 360(full circle) - 60 - 60 = 240}
                \end{split}
            \end{equation*}

            \item
                \begin{enumerate}
                    \item The 2 trianlges are similar because their angles are the same (60 degrees).
                    \item The 2 trianlges are not congruent because their sides are not the ame length.
                \end{enumerate}
            \item TODO
                \begin{equation*}
                    \begin{split}
                    \end{split}
                \end{equation*}
        \end{enumerate}

    %question 7
    \pagebreak
    \item
        \begin{enumerate}
            \item 
            \begin{equation*}
                \begin{split}
                    \text{Diagram A}\\
                    \text{4 + 10 + 3rd side}&= 26\\
                    \text{3rd side}&=26-4-10\\
                    &=12\\
                    \text{Diagram B}\\
                    \text{9 + 10 + 3rd side}&= 26\\
                    \text{3rd side}&=26-9-10\\
                    &=7
                \end{split}
            \end{equation*}
            
            \item
            \begin{equation*}
                \begin{split}
                    \text{4 or 9 or 10}
                \end{split}
            \end{equation*}

            \item
            \begin{enumerate}
                \item
                \begin{equation*}
                    \begin{split}
                        19cm^2
                    \end{split}
                \end{equation*}
    
                \item
                \begin{equation*}
                    \begin{split}
                        TODO
                    \end{split}
                \end{equation*}
    
                \item TODO
                \begin{equation*}
                    \begin{split}
                        x=8
                    \end{split}
                \end{equation*}
            \end{enumerate}

            \item Disprove pythagoras theorem.
            \begin{equation*}
                \begin{split}
                    a^2+b^2&\neq c^2\\
                    5^2+10^2&\neq11^2\\
                    25+100&\neq121
                \end{split}
            \end{equation*}

            \item Construct a perpendicular line from the bottom side to the top angle in order to find the height of the triangle. Then use the area of a triangle formula.
            \begin{equation*}
                \begin{split}
                    height&=\sqrt{8^2-5^2}\\
                    &=\sqrt{64-25}\\
                    &=\sqrt{39}\\
                    \\            
                    Area&=\frac{1}{2}*base*height\\
                    &=\frac{1}{2}*10*\sqrt{39}\\
                    &=5\sqrt{39}
                \end{split}
            \end{equation*}
        \end{enumerate}

    %question 8
    \pagebreak
    \item
        \begin{enumerate}
            \item 1
            
            \item Roll a 7 or an 8, representing 2 out of 6 possible outcomes.
            \begin{equation*}
                \begin{split}
                    \text{P(greater than 6)}&=\frac{2}{6}\\
                    &=\frac{1}{3}
                \end{split}
            \end{equation*}

            \item First find the probability of rolling an odd number in 1 roll. 4 out of 6 possible outcomes are odd.
            \begin{equation*}
                \begin{split}
                    \text{P(odd)}&=\frac{4}{6}\\
                    &=\frac{2}{3}\\
                \end{split}
            \end{equation*}
            Then multiple this probability by the number of rolls.
            \begin{equation*}
                \begin{split}
                    60*\frac{2}{3}=40
                \end{split}
            \end{equation*}
            \item \ 
                \begin{center}
                    \begin{tabular}{|c|c|} 
                        \hline
                        Description&Term\\
                        \hline
                        The set of all possible outcomes of an experiment&Sample Space\\
                        One possible result of an experiement&Outcome\\
                        A subset of the sample space - a collection of one or more outcomes&Event\\
                        \hline
                    \end{tabular}
                \end{center}
        \end{enumerate}

    %question 9
    \pagebreak
    \item
        \begin{enumerate}
            \item (i), (ii), (iii)
            \begin{tikzpicture}[grow=right, sloped]
                \node[bag] {}
                    child {
                        node[bag] {T}        
                        child {
                            node[end, label=above:{T}] {}
                            child {
                                node[end, label=right:{T, TTT, 0}] {}
                            }
                            child {
                                node[end, label=right:{H, TTH, 1}] {}
                            }
                        }
                        child {
                            node[end, label=above:{H}] {}
                            child {
                                node[end, label=right:{T, THT, 1}] {}
                            }
                            child {
                                node[end, label=right:{H, THH, 2}] {}
                            }
                        }
                    }
                    child {
                        node[bag] {H}        
                        child {
                            node[end, label=above:{T}] {}
                            child {
                                node[end, label=right:{T, HTT, 1}] {}
                            }
                            child {
                                node[end, label=right:{H, HTH, 2}] {}
                            }
                        }
                        child {
                            node[end, label=above:{H}] {}
                            child {
                                node[end, label=right:{T, HHT, 2}] {}
                            }
                            child {
                                node[end, label=right:{H, HHH, 3}] {}
                            }
                        }      
                    };
                \end{tikzpicture}
            
            \item \ 
            \begin{center}
                \begin{tabular}{|c|c|c|c|c|} 
                    \hline
                    Number of Heads (H)&0&1&2&3\\
                    \hline
                    Probability&$\frac{1}{8}$&$\frac{3}{8}$&$\frac{3}{8}$&$\frac{1}{8}$\\
                    \hline
                \end{tabular}
            \end{center}

            \item
                \begin{enumerate}
                    \item To flip no heads after 8 coin flips is very unlikely. The can only happen in the event that she flips all tails.
                    \begin{equation*}
                        \begin{split}
                            1
                        \end{split}
                    \end{equation*}

                    \item There are 8 outcomes with exactly 1 head. A head on the first flip only, or a tail on the second flip only or the thirs flip only, and so on.
                    \begin{equation*}
                        \begin{split}
                            8
                        \end{split}
                    \end{equation*}
                \end{enumerate}
            \item
                \begin{equation*}
                    2*2*2*2*2*2*2*2=2^8=256
                \end{equation*}
        \end{enumerate}

    %question 10
    \pagebreak
    \item
        \begin{enumerate}
            \item Use the circumference of a circle formula.
            \begin{equation*}
                \begin{split}
                    C&=2 \pi r = \pi d\\
                    &=\pi(8)\\
                    &=25.1
                \end{split}
            \end{equation*}
            
            \item The rubber track goes around the circumference of a semi-circle on both ends,
            and over 3 diameters of the circle on the top and the bottom.
            \begin{equation*}
                \begin{split}
                    length&=(semicircle)+(semicircle)+(top\ straight)+(bottom\ straight)\\
                    &=(\pi r)+(\pi r)+(3d)+3d)\\
                    &=2(\pi r)+(6d)\\
                    &=25.1+48\\
                    &=73.1
                \end{split}
            \end{equation*}

            \item We need to divide the total length of track by the length of circumference of one wheel.
            \begin{equation*}
                \begin{split}
                    turns&=\frac{length\ of\ track}{circumference\ of\ wheel}\\
                    &=\frac{73.1}{25.1}\\
                    &=2.9124\\
                    &=2\ full\ turns
                \end{split}
            \end{equation*}
        \end{enumerate}

    %question 11
    \pagebreak
    \item Divide the shape into 2 smaller rectangles.
    The smaller one on the left having dimensions s and 6-s.
    And the larger right one having dimensions 6 and 10-s.
        \begin{enumerate}
            \item 
            \begin{equation*}
                \begin{split}
                    Area_{\ total}&=Area_{\ smaller}+Area_{\ bigger}\\
                    &=(s)(6-s)+(6)(10-s)\\
                    &=(6s-s^2)+(60-6s)\\
                    &=60-s^2
                \end{split}
            \end{equation*}
            
            \item
            \begin{equation*}
                \begin{split}
                    Perimeter&=10+6+(10-s)+(6-s)+s+s\\
                    &=10+10+6+6\\
                    &=32\text{, since the s's cancel, the value doesn't matter}
                \end{split}
            \end{equation*}
        \end{enumerate}

    %question 12
    \pagebreak
    \item
        \begin{enumerate}
            \item TODO
            \begin{equation*}
                \begin{split}
                \end{split}
            \end{equation*}
        \end{enumerate}

    %question 13
    \pagebreak
    \item
        \begin{enumerate}
            \item 
            \begin{equation*}
                \begin{split}
                    sin(20)&=\frac{6}{x}\\
                    (x)sin(20)&=6\\
                    x&=\frac{6}{sin(20)}\\
                    &=17.54
                \end{split}
            \end{equation*}
            
            \item
            \begin{equation*}
                \begin{split}
                    cos(Y)&=\frac{a}{c}\\
                    sin(Y)&=\frac{b}{c}\\
                    cos(Y)+sin(Y)&=\frac{a}{c}+\frac{b}{c}\\
                    &=\frac{a+b}{c}\\
                    if\ a+b>c,\ th&en\ \frac{a+b}{c}>1\text{, since numerator $>$ denominator}
                \end{split}
            \end{equation*}
        \end{enumerate}

    %question 14
    \pagebreak
    \item
        \begin{equation*}
            \begin{split}
                V_{\ cylinder}&=\pi r^2h=\pi r^2 10=10\pi r^2\\
                V_{\ cone}&=\pi r^2\frac{h}{3}\\
                V_{\ cone}&=V_{\ cylinder}*90\%\\
                \pi r^2\frac{h}{3}&=10\pi r^2*90\%\\
                \pi r^2\frac{h}{3}&=9\pi r^2\\
                r^2\frac{h}{3}&=9r^2\\
                \frac{h}{3}&=9\\
                h&=27\\
                increase&=\frac{new\ height-old\ height}{old\ height}\\
                &=\frac{27-10}{10}\\
                &=\frac{17}{10}\\
                &=170\%\ increase
            \end{split}
        \end{equation*}
    \end{enumerate}
\end{document}