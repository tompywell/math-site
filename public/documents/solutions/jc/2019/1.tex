\documentclass[11pt]{article}
\usepackage{amsmath}
\usepackage{multicol}
\input longdiv.tex
\usepackage{polynom}

\topmargin -.5in
\textheight 9in
\oddsidemargin -.25in
\evensidemargin -.25in
\textwidth 7in

\begin{document}

\author{Tom Pywell}
\title{2019 Junior Certificate Higher Level Paper One}
\maketitle

\medskip

\begin{enumerate}
    %question 1
    \pagebreak
    \item
        \begin{enumerate}
            \item If 2/5 of the students do Business Studies, then the remaining amount of 3/5 don't.
                \begin{equation*}
                \frac{3}{5} * 85 = 51
                \end{equation*}
            \item To find 26 as a percentage of 85, divide 26 by 85 and multiply by 100.
                \begin{equation*}
                    \frac{26}{85} * \frac{100}{1} = 30.6\%
                \end{equation*}
            \item
                Let's imagine we have 60 students and 4 teachers, since 60:4 = 15:1.
                Adding a single teacher to the total makes the new ratio 60:5 = 12:1.
                In all scenarios the left-hand-side of the new ratio will be $<$ 15.
                
                \begin{equation*}
                    a < 15
                \end{equation*}
        \end{enumerate}
    
    %question 2
    \pagebreak
    \item
        \begin{enumerate}
            \item Factors are numbers that multiply to give a desired number.
                \begin{equation*}
                    \begin{split}
                        1*45&=45\\
                        3*15&=45\\
                         5*9&=45\\
                    \end{split}
                \end{equation*}
                \begin{equation*}
                3,5,9,15
                \end{equation*}
                
            \item
                A composite number is a non-prime positive integer.
                A prime number is a number whose only factors are itself and 1.
                A square number is the product of any integer multiplied by itself.
                n is a prime number since it's only factors are itself and 1.
            
            \item If 7 is the only other factor of p, then it's co-factor must also be 7. p must be a square number.
                \begin{equation*}
                    \begin{split}
                    1*p&=7*7\\
                    p&=49
                    \end{split}
                \end{equation*}
                
            \item For this question we'll look for common factors in the number 12 and 8.
                \begin{equation*}
                    \begin{split}
                        (2)(6k+4)=12k+8\\
                        (4)(3k+2)=12k+8\\
                        \\
                        2, 6k+4, 4, 3k+2
                    \end{split}
                \end{equation*}
        \end{enumerate}
    
    %question 3
    \pagebreak
    \item
        \begin{enumerate}
            \item To make 56,000,000 somewhere between 1 and 10, we'd need to move the decimal place 7 places to the left, leaving us with 5.6.
            Since moving the decimal place left divides the number by 10, moving the decimal point 7 times will divide the number by 10, 7 times. To keep the number's original value, we have to multiply (undo the division) by 10 7 times.
            \begin{equation*}
                \begin{split}
                56,000,000&=56,000,000*10^0\\
                &=5,600,000*10^1\\
                &=560,000*10^2\\
                &=56,000*10^3\\
                &=5,600*10^4\\
                &=560*10^5\\
                &=56*10^6\\
                &=5.6*10^7
                \end{split}
            \end{equation*}
            
            \item Similar to previous question, but in the opposite direction.
            \begin{equation*}
                \begin{split}
                    0.0075&=0.0075*10^0\\
                    &=0.075*10^-1\\
                    &=0.75*10^-2\\
                    &=7.5*10^-3
                \end{split}
            \end{equation*}
            
            \item Let's convert his speed from km/hr to m/s and see how long he'd travel in 0.5s.
            \begin{equation*}
                \begin{split}
                    &=90 km/h\\
                    90*1000&=90,000 m/h\\
                    90,000/60&= 150 m/m\\
                    1500/60&= 25m/s\\
                    25 * 0.5&= 12.5m
                \end{split}
            \end{equation*}
            If Lewis was travelling 25m/s, he would cover 12.5 metres in 0.5 seconds.
        \end{enumerate}

    %question 4
    \pagebreak
    \item
        \begin{enumerate}
            \item Gross income is income before any deductions/tax have been subtracted.
                \begin{enumerate}
                    \item We must deduct the pension contributions from the gross income to find the taxable income.
                        \begin{equation*}
                            \begin{split}
                                \text{pension contributions}&=52460*8.5\%\\
                                &=4459.1\\
                                \text{taxable income }&=\text{gross income - pension contributions}\\
                                &=52460-4459.1\\
                                &=48000.9\\
                            \end{split}
                        \end{equation*}
                        
                    \item We apply the tax rates the the relevant taxable amounts, then subtract tax credits from their sum.
                        \begin{equation*}
                            \begin{split}
                                34000\ @\ 20\%&=6800\\
                                (48000.9-34000)\ @\ 40\%&=5600.36\\
                                \text{gross tax }&=6800+5600.36\\
                                &=12400.36\\
                                \text{net tax }&=\text{ gross tax - tax credits }\\
                                &=12400.36-4200\\
                                &=8200.36\\
                                \text{net income }&=\text{ taxable income - net tax}\\
                                &=48000.9-8200.36\\
                                &=39800.54
                            \end{split}
                        \end{equation*}
                \end{enumerate}
                \item Compound interest, we can calculate it month by month, 3 times, or all at once using compound interest formula.
                    \begin{equation*}
                        \begin{split}
                            420*102\%&=428.4\\
                            428.4*102\%&=436.968\\
                            436.968*102\%&=445.70736=446\\
                            &or\\
                            420*(1+0.02)^3&=445.70736=446
                        \end{split}
                    \end{equation*}
                \item In this question we're working with a known value fo 90\% and working backwards to 100\%.
                    \begin{equation*}
                        \begin{split}
                            12150&=90\%\\
                            \frac{12150}{90\%}&=100\%\\
                            &=13500
                        \end{split}
                    \end{equation*}
        \end{enumerate}

    %question 5
    \pagebreak
    \item
        \begin{enumerate}
            \item Find an element that is in all 3 sets.
                \begin{equation*}
                    \begin{split}
                        A&=\{2,4,6,8,10,\underline{12},...\}\\
                        B&=\{3,6,9,\underline{12},15,18,...\}\\
                        C&=\{4,8,\underline{12},16,20,...\}\\
                        &\text{12 is in all 3 sets}
                    \end{split}
                \end{equation*}
            \item C is a subset of A because any multiple of 4 is also a mulitple of 2.
            \item
                There are no elements in C/(A$\cup$B), since any would also be in A.\\
                There are no elements in (C$\cap$B)/A for the same reason.
            \item
                From top-left, to bottom-right.
                \begin{equation*}
                    \begin{split}
                        &2, 6, 3\\
                        &4, 12, ...\\
                        &...
                    \end{split}
                \end{equation*}
        \end{enumerate}     
        
        
    %question 6
    \pagebreak
    \item
        \begin{enumerate}
            \item
                TODO
            \item
                \begin{center}
                    \begin{tabular}{ |c|c|c| } 
                        \hline
                        distance & Poppy time & Ella time \\ 
                        \hline
                        1 & 5 & 4 \\ 
                        2 & 10 & 8 \\ 
                        3 & 17 & 14 \\
                        4 & 24 & 20 \\
                        5 & 30 & 26 \\
                        \hline
                    \end{tabular}
               \end{center}
            \item
                To prove a sequence in non quadratic, we can show that the second differences between adjacent terms is not a constant. Below we can see the second difference is not a constant (i.e. it changes).
                \begin{center}
                    \begin{tabular}{ |c|c|c| } 
                        \hline
                        Poppy Time & 1st difference & 2nd difference \\ 
                        \hline
                        5 & - & - \\ 
                        10 & +5 & - \\ 
                        17 & +7 & +2 \\
                        24 & +7 & +0 \\
                        30 & +6 & -1 \\
                        \hline
                    \end{tabular}
               \end{center}
            \item
                \begin{equation*}
                    \begin{split}
                        speed&=\frac{distance}{time}\\
                        &=\frac{5}{26}km/min\\
                        \frac{5}{26}km/min&=(\frac{5}{26}*60)km/hr\\
                        &=11.54km/hr
                    \end{split}
                \end{equation*}

            \item
                As the slope of the graph increases, this mean per kilometer Ella travelled, it took more time than before.
                If travelling distance takes more time, she must be travelling slower. Ella's speed decreased.
            \item 
                At part C, as total time increases, distance travelled stays a constant. This must mean Ciarán is not moving.
            \item
                No idea, I would have said Brendan was incorrect.
        \end{enumerate}
    
    %question 7
    \pagebreak
    \item
        \begin{enumerate}
            \item
                \begin{enumerate}
                    \item
                        The set of natural numbers N is all positive whole numbers. Zero is neither positive nor negative, therefore it is not a natural number.\\
                        \begin{equation*}
                            N = \{1, 2, 3, 4, 5, 6, 7, 8, 9, ...\}
                        \end{equation*}
                    \item
                        The set of integers Z is all whole numbers.
                        \begin{equation*}
                            Z = \{..., -3, -2, -1, 0, 1, 2, 3,...\}
                        \end{equation*}
                    \end{enumerate}
            \item
                TODO
            \item
                To solve a double/compound inequality, we solve both individually and combine their results.
                \begin{minipage}{0.5\textwidth}
                        \begin{equation*}
                            \begin{split}
                                -7&<8-3g\\
                                3g&<8+7\\
                                3g&<15\\
                                g&<5
                            \end{split}
                        \end{equation*}
                \end{minipage}%
                \begin{minipage}{0.5\textwidth}
                        \begin{equation*}
                            \begin{split}
                                8-3g&\leq11\\
                                -3g&\leq11-8\\
                                -3g&\leq3\\
                                3g&\geq-3\\
                                g&\geq-1
                            \end{split}
                        \end{equation*}
                \end{minipage}
                \begin{equation*}
                    \begin{split}
                        -1\leq g<5
                    \end{split}
                \end{equation*}
                We generally want our answer with smallest number on the left, followed by variable in the middle, followed by larger number on the right.
        \end{enumerate}

    %question 8
    \pagebreak
    \item The total number of people could be calculated by
        \begin{equation}
            52+(184-n)+2n+125=361+n
        \end{equation}
        To maximise the total, 361 + n, we'll need to maximize n.
        The smallest value 188-n can take is 0, since any larger and we'd have a negative number of elements in the intersection set, which isn't possible.
        So the largest value n could take is 188, and the argest value our total can take is 361+n, where n=188.
        \begin{equation*}
            361+188=549
        \end{equation*}

    %question 9
    \pagebreak
    \item
        \begin{enumerate}
            \item
                \begin{enumerate}
                    \item 
                        Every 3rd term is 4. Since 12 is a multiple of 3, the 12th term will be 4.
                    \item
                        Every 3rd term is 4. Since 99 is a multiple of 3, the 99th term will be 4.
                        100th term will be 3, since 3 always comes after 4 in the sequence.
                \end{enumerate}
            \item 
                To find the nth term of the sequence, calculate n\%3, the remainder when n is divided by 3.
                The nth term takes the corresponding place in the sequence. If the remainder is 0, then the nth is the same as the 3rd term. e.g.
                \begin{equation*}
                    \begin{split}
                        100\%3&=1, \text{100th term is the same as the 1st term}\\
                        99\%3&=0, \text{99th term is the same as the 0th, or 3rd term}\\
                        98\%3&=2, \text{98th term is the same as the 2nd term}
                    \end{split}
                \end{equation*}
            \item
                \{8, 5, 10, 6, 4\}
                \begin{equation*}
                    \begin{split}
                        \text{1st term}&=8\\
                        \text{8 is even, }(8+2)*\frac{1}{2}&=5\\
                        \text{5 is odd, }5*2&=10\\
                        \text{10 is even, }(10+2)*\frac{1}{2}&=6\\
                        \text{6 is even, }(6+2)*\frac{1}{2}&=4\\
                    \end{split}
                \end{equation*}

            \item Below we can see that Ahmed's sequence remains a constant. Apparently this is unusual.
                \begin{equation*}
                    \begin{split}
                        \text{1st term}&=2\\
                        \text{2 is even, }(2+2)*\frac{1}{2}&=2\\
                        \text{2 is even, }(2+2)*\frac{1}{2}&=2\\
                        \text{2 is even, }(2+2)*\frac{1}{2}&=2
                    \end{split}
                \end{equation*}

            \item
                We're looking for an odd number and an even number, x and y,
                where the result of calculating the next term results in 86.
                \begin{minipage}{0.5\textwidth}
                    \begin{equation*}
                        \begin{split}
                            &\text{previous was odd}\\
                            x*2&=86\\
                            x&=43
                        \end{split}
                    \end{equation*}
                \end{minipage}%
                \begin{minipage}{0.5\textwidth}
                    \begin{equation*}
                        \begin{split}
                            &\text{previous was even}\\
                            (y+2)*\frac{1}{2}&=86\\
                            (y+2)&=172\\
                            y&=170
                        \end{split}
                    \end{equation*}
                \end{minipage}

            \item
                We're going to have to do a little bit of inspection to know whether numbers in terms of k are odd or even.
                \begin{equation*}
                    \begin{split}
                        \text{1st term}&=k\\
                        \text{k is odd, }(k*2)&=2k\\
                        \text{2k is even, since 2 times any number is even, }(2k+2)*\frac{1}{2}&=k+1\\
                        \text{k + 1 is even, since k is odd, }((k+1)+2)*\frac{1}{2}&=\frac{k+3}{2}\\
                        \\
                        k,\ 2k,\ k+1,\ \frac{k+3}{2}
                    \end{split}
                \end{equation*}
                
        \end{enumerate}

    %question 10
    \pagebreak
    \item
        TODO

    %question 11
    \pagebreak
    \item
        \begin{equation*}
            \begin{split}
                \frac{3x+5}{2}+\frac{x-4}{3}&=16\\
                \frac{3x+5}{2}+\frac{x-4}{3}&=\frac{16}{1}\text{, common denominator is 6}\\
                \frac{(2)(3)(3x+5)}{2}+\frac{(2)(3)(x-4)}{3}&=\frac{(2)(3)(16)}{1}\text{, cancel common factors}\\
                (3)(3x+5)+(2)(x-4)&=(2)(3)(16)\\
                (9x+15)+(2x-8)&=96\\
                11x+7&=96\\
                11x&=89\\
                x=\frac{89}{11}
            \end{split}
        \end{equation*}

    %question 12
    \pagebreak
    \item
        \begin{enumerate}
            \item Notice it matches difference of 2 squares pattern
                \begin{equation*}
                    \begin{split}
                        a^2-16n^2\\
                        (a-4)(a+4)
                    \end{split}
                \end{equation*}

            \item
                \begin{enumerate}
                    \item
                        \begin{equation*}
                            \begin{split}
                                8x^2+45x-18\\
                                (x+6)(8x-3)
                            \end{split}
                        \end{equation*}
                    \item
                        \begin{equation*}
                            \begin{split}
                                x^2+7x+6\\
                                (x+6)(x+1)
                            \end{split}
                        \end{equation*}
                \end{enumerate}

            \item
                TODO   
        \end{enumerate}

    %question 13
    \pagebreak
    \item
        \begin{enumerate}
            \item
                \begin{equation*}
                    \begin{split}
                        P&=36(70)+62(0.65)+1800\\
                        &=4360.3
                    \end{split}
                \end{equation*}

            \item
                \begin{equation*}
                    \begin{split}
                        4360.3+8&=36(70)+62(h)+1800\\
                        4368.3&=2560+62h+1800\\
                        48.3&=62h\\
                        h&=\frac{48.3}{62}\\
                        h&=0.78
                    \end{split}
                \end{equation*}

            \item
                TODO
            \item
            \begin{equation*}
                \begin{split}
                    60w+80h-15f-1300&=36w+62h+1800\\
                    80h-62h&=36w+1800-60w+15f+1300\\
                    18h&=-24w+15f+3100\\
                    h&=\frac{-24w+15f+3100}{18}
                \end{split}
            \end{equation*}
        \end{enumerate}

    %question 14
    \pagebreak
    \item
        \begin{enumerate}
            \item
                \begin{minipage}{0.5\textwidth}
                    \begin{equation*}
                        \begin{split}
                            (x^2-2x)-(6x-3)\\
                            x^2-2x-6x+3\\
                            x^2-8x+3
                        \end{split}
                    \end{equation*}
                \end{minipage}%
                \begin{minipage}{0.5\textwidth}
                    \begin{equation*}
                        \begin{split}
                            (4x^2+3x)-(x^2-2x)\\
                            4x^2+3x-x^2+2x\\
                            3x^2+5x
                        \end{split}
                    \end{equation*}
                \end{minipage}
            \item If the sequence is linear then the differences between terms are equal.
                \begin{equation*}
                    \begin{split}
                        x^2-8x+3&=3x^2+5x\\
                        x^2-8x+3-3x^2-5x&=0\\
                        -2x^2-13x+3&=0\\
                        2x^2+13x-3&=0
                    \end{split}
                \end{equation*}
            \item Quadratic factorization to decimal places. Use "-b" formula.
                \begin{equation*}
                    \begin{split}
                        &2x^2+13x-3=0\text{ is in the form }ax^2+bx-c=0\text{ so a=2, b=13, c=-3}\\
                        x&=\frac{-b\pm\sqrt{b^2-4ac}}{2a}\\
                        &=\frac{-13\pm\sqrt{13^2-4(2)(-3)}}{2(2)}\\
                        &=\frac{-13\pm\sqrt{169+24}}{4}\\
                        &=\frac{-13+\sqrt{193}}{4}\text{ and }\frac{-13-\sqrt{193}}{4}\\
                        &=0.223\text{ and }-6.723
                    \end{split}
                \end{equation*}
        \end{enumerate}
    \end{enumerate}
\end{document}